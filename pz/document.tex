
\documentclass[a4paper,12pt, fleqn]{article}
% fleqn - Выравнивание формул по левому краю


%%% Работа с русским языком
\usepackage{cmap}					% поиск в PDF
\usepackage{mathtext} 				% русские буквы в формулах
\usepackage[T2A]{fontenc}			% кодировка
\usepackage[utf8]{inputenc}			% кодировка исходного текста
\usepackage[english,russian]{babel}	% локализация и переносы

%%% Дополнительная работа с математикой
\usepackage{amsmath,amsfonts,amssymb,amsthm,mathtools} % AMS
\usepackage{icomma} % "Умная" запятая: $0,2$ --- число, $0, 2$ --- перечисление

%%% Работа с цветом
\usepackage[usenames]{color}
\usepackage{colortbl}

%% Номера формул
%\mathtoolsset{showonlyrefs=true} % Показывать номера только у тех формул, на которые есть \eqref{} в тексте.
%\usepackage{leqno} % Нумерация формул слева

%% Свои команды
\DeclareMathOperator{\sgn}{\mathop{sgn}}

%% Перенос знаков в формулах (по Львовскому)
\newcommand*{\hm}[1]{#1\nobreak\discretionary{}
	{\hbox{$\mathsurround=0pt #1$}}{}}

%%% Работа с картинками
\usepackage{graphicx}  % Для вставки рисунков
\graphicspath{{images/}{WinLogic/}}  % папки с картинками
\setlength\fboxsep{3pt} % Отступ рамки \fbox{} от рисунка
\setlength\fboxrule{1pt} % Толщина линий рамки \fbox{}
\usepackage{wrapfig} % Обтекание рисунков текстом

%%% Работа с таблицами
\usepackage{array,tabularx,tabulary,booktabs} % Дополнительная работа с таблицами
\usepackage{longtable}  % Длинные таблицы
\usepackage{multirow} % Слияние строк в таблице

%%% Теоремы
\theoremstyle{plain} % Это стиль по умолчанию, его можно не переопределять.
%\newtheorem{theorem}{Теорема}[section]
%\newtheorem{proposition}[theorem]{Утверждение}

\theoremstyle{definition} % "Определение"
%\newtheorem{corollary}{Следствие}[theorem]
%\newtheorem{problem}{Задача}[section]

\theoremstyle{remark} % "Примечание"
\newtheorem*{nonum}{Решение}

%%% Программирование
\usepackage{etoolbox} % логические операторы


%%% Функции, свяханные с этим документом
\newcommand{\DocumentNum}{
	RU.17701729.506900-01 ТЗ 01-1
}

\newcommand{\Changing}{
	\ensuremath{
		$$
		\begin{array}{|c|c|c|c|c|}
		\hline
		&&&&\\
		\hline
		\text{Изм.}&\text{Лист}&\text{№ докум.}&\text{Подп.}&\text{Дата}\\
		\hline
		\text{\DocumentNum}&&&&\\
		\hline
		\text{Инв. № подл.}&\text{Подп. и дата}&\text{Взам. инв. №}&\text{Инв. № дубл.}&\text{Подп. и дата}\\
		\hline
		\end{array}
		$$
	}
}


%%% Страница
%\usepackage{extsizes} % Возможность сделать 14-й шрифт
\usepackage{geometry} % Простой способ задавать поля
\geometry{top=20mm}
\geometry{bottom=20mm}
\geometry{left=30mm}
\geometry{right=10mm}
%
\usepackage{fancyhdr} % Колонтитулы
\pagestyle{fancy}
\renewcommand{\headrulewidth}{0mm}  % Толщина линейки, отчеркивающей верхний колонтитул
\lfoot{}%Нижний левый
\rfoot{}%Нижний правый
\rhead{Вариант 12}%Верхний правый
\chead{Торилов Дмитрий}%Верхний в центре
\lhead{}%Верхний левый
\cfoot{\thepage} % По умолчанию здесь номер страницы


\usepackage{setspace} % Интерлиньяж
\onehalfspacing % Интерлиньяж 1.5
%\doublespacing % Интерлиньяж 2
%\singlespacing % Интерлиньяж 1

\usepackage{lastpage} % Узнать, сколько всего страниц в документе.

\usepackage{soulutf8} % Модификаторы начертания

\usepackage{hyperref}
\usepackage[usenames,dvipsnames,svgnames,table,rgb]{xcolor}
\definecolor{my-color}{HTML}{226634}
\hypersetup{				% Гиперссылки
	unicode=true,           % русские буквы в раздела PDF
	pdftitle={Пояснительная записка},   % Заголовок
	pdfauthor={Автор},      % Автор
	pdfsubject={КДЗ по программированию},      % Тема
	pdfcreator={Создатель}, % Создатель
	pdfproducer={Производитель}, % Производитель
	pdfkeywords={keyword1} {key2} {key3}, % Ключевые слова
	colorlinks=true,       	% false: ссылки в рамках; true: цветные ссылки
	linkcolor=black,          % внутренние ссылки
	citecolor=my-color,        % на библиографию
	filecolor=magenta,      % на файлы
	urlcolor=cyan           % на URL
}

%\renewcommand{\familydefault}{\sfdefault} % Начертание шрифта

\usepackage{csquotes} % Инструменты для ссылок

%\usepackage[bibencoding=utf8]{biblatex}
%\addbibresource{bib1.bib}
%\addbibresource{GOST.bib}

% Обещают Times New Roman
\renewcommand{\rmdefault}{ftm}


\usepackage{multicol} % Несколько колонок

\author{Торилов Дмитрий}
\title{Пояснительная записка}
\date{\today}


%\textbf{полужирный шрифт}
%\textit{курсивом}

\newcommand{\mc}[1]{%
	%\underline{#1}%\colorbox{YellowGreen}{#1}%
	%\ensuremath{
	%	{\textcolor{blue}{\textbf{#1}}}
	%}
	%\boldsymbol{#1}
	\cellcolor{YellowGreen}#1
} % blue color highlighting
%\textcolor{blue}{\textbf{Синий}}

%%% Нумeрация формул зависит от раздела
\numberwithin{equation}{section}


%%% Программирование на LaTeX
\usepackage{forloop}


\newcommand{\Ssign}{\underline{\hspace{8em}}}


\newcommand{\PeopleField}[1]{
	\vbox to 10em{
		#1\\
		<<\underline{\hspace{1.8em}}>>
		\underline{\hspace{10.5em}}
		2018 г.
	}
}


\newcommand{\StorageTable}{
	$$
	\begin{tabular}{|m{5mm}|p{7mm}|}
	\hline
	\rotatebox{90}{
		Дата и подп.
	}&\rule{0mm}{35mm}\\
	\hline
	\rotatebox{90}{
		Инв. № дубл.
	}&\rule{0mm}{25mm}\\
	\hline
	\rotatebox{90}{
		Взвм. инв. №
	}&\rule{0mm}{25mm}\\
	\hline
	\rotatebox{90}{
		Дата и подп.
	}&\rule{0mm}{35mm}\\
	\hline
	\rotatebox{90}{
		Инв. № подл.
	}&\rule{0mm}{25mm}\\
	\hline
	\end{tabular}
	$$
}

%%% Содержание
\renewcommand{\thesection}{\arabic{section}.}
\renewcommand{\thesubsection}{\arabic{section}.\arabic{subsection}.}
\renewcommand{\thesubsubsection}{\arabic{section}.\arabic{subsection}.\arabic{subsubsection}.}
\makeatletter
\renewcommand{\l@section}{\@dottedtocline{1}{0.5em}{1.5em}}
\renewcommand{\l@subsection}{\@dottedtocline{1}{1.5em}{2.5em}}
\renewcommand{\l@subsubsection}{\@dottedtocline{1}{3em}{3em}}
\makeatother


\begin{document}

\begin{titlepage}
		
	\large
	\begin{center}
		НАЦИОНАЛЬНЫЙ ИССЛЕДОВАТЕЛЬСКИЙ УНИВЕРСИТЕТ\\
		<<ВЫСШАЯ ШКОЛА ЭКОНОМИКИ>>\\
		Дисциплина: <<Программирование>>\\[10em]
		
		Контрольное домашнее задание\\
		3 модуль\\
		Вариант 12\\[7em]	
	\end{center}
	
	\begin{flushright}
		\PeopleField{
			Выполнил\\
			Студент группы БПИ173\\
			\Ssign /Д. М. Торилов/
		}
		
		Преподаватель: Максименкова О.В.,\\
		старший преподаватель\\
		департамента\\
		программной инженерии\\
		факультета компьютерных наук\\[7em]
	\end{flushright}
	
	\begin{center}
		Москва 2017	
	\end{center}
	
	\normalsize
	\newpage

\end{titlepage}

% Оглавление
\setcounter{page}{2}
\tableofcontents
\newpage

\section{Условие задачи}
Программа контрольного домашнего задания (КДЗ) должна представлять собой небольшую информационно-справочную систему (ИСС), основанную на файлах. В стандартном файле содержатся данные о землетрясениях. Данные из него загружаются в основную таблицу ИСС.\\[1em]
Далее следует описание задания варианта \textnumero 12:\\[1em]
Для представления данных о землетрясении использовать класс EarthQuake. Координаты землетрясения представлять объектом структуры. Класс QuakeInfo связан с объектами EarthQuake отношением агрегации и позволяет получать списки землетрясений, сгруппированные по количеству уловивших их станций; списки землетрясений с максимальной магнитурой; землетрясение произошедшее на минимальной и максимальной глубине. Модифицировать интерфейс так, чтобы указанные данные можно было отобразить.


\section{Функции разрабатываемого приложения}

\subsection{Варианты использования}
Данная ИСС может быть использована для проведения исследований в области изучения землетрясений, в том числе в научных и образовательных целях.

\subsection{Описание интерфейса пользователя}


\section{Структура приложения}
Приложение реализовано с использованием паттерна Model-View-Controller.
\subsection{Диаграмма классов}

\subsection{Описание классов, их полей и методов}
(Кратко описывается назначение классов и других типов данных, введённых в программе. Перечисляются все члены типов и описываются естественным языком. Указываются виды отношений между классами.)


\section{Распределение исходного кода по файлам проекта}
Содержит не коды из файлов, а краткое описание на естественном языке, какой функционал реализуется тем или иным кодом)


\section{Контрольный пример и описание результатов}
(Контрольный пример – это аккуратно описанная последовательность действий, позволяющая проверить корректность работы функций программы по шагам. Это предполагает для каждого шага наличие входных и выходных данных или состояния интерфейса и т.п.)


\section{Текст (код) программы}
(только части написанные вручную).


\section{Список литературы}
оформленный по ГОСТ 7.05-2008 (в тексте пояснительной записки должны быть ссылки на цитируемую литературу). 


\end{document}